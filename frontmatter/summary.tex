\chapter*{Summary}
\addcontentsline{toc}{chapter}{Summary}
Much like the oil crisis in the 1980s, the aviation industry is once again considering more environmentally sustainable propulsion options due to concerns around climate change. More fuel efficient propulsion methods form the pillar on which environmentally sustainable aviation can be built, since it is the propulsion system that emits harmful greenhouse gasses. The propulsion system consists of an energy carrier, currently kerosene, and an energy to thrust converter, often a jet engine. Although efficiency gains are being made, using kerosene and jet engines will prove infeasible in the long run due to global warming concerns. Replacing the jet engine with a propeller offers an attractive solution since propeller shaft power can be supplied by batteries or fuel cells. Furthermore, propellers are relatively efficient due to their infinite bypass ratio. Additionally, a surge in demand for Urban Air Mobility~\cite{Antcliff2019a} incentives propeller optimisation studies.

Aircraft are a combination of complex and interacting systems. For this reason it is important to consider interactions between wings and propellers when designing either. Propeller-wing optimisation is therefore an increasingly important topic. Propeller-wing optimisation literature is scarce, likely due to the complexity of a coupled propeller-wing system. Optimising a propeller-wing system is possible with high-fidelity simulations but often takes a substantial amount of time. The aim of this research is to address the lack of coupled propeller-wing aerostructural optimisation. The knowledge gap is addressed by designing a novel coupled propeller wing framework that is suited for computationally efficient optimisation studies. Furthermore, the optimisation framework will be modular such that it can be easily extended. The ability to expand the framework increases the scope and impact of this research.\\
\\
Computational efficiency for gradient-based optimisation is determined by the model choice, the gradient assessment method, and the optimisation architecture. This research project encompasses the former two and leaves the optimisation algorithm out of its scope: The optimisation algorithm used is SNOPT~\cite{gill2005snopt}, a Sequential Quadratic Programming algorithm. Modelling choices play an important role since computational efficiency is often inversely related to accuracy. Therefore, the models should be carefully chosen such that all are of the desired level of fidelity and computational expense. Additionally, the gradient assessment method is arguably the most important choice to determine the accuracy and computational efficiency of an optimisation architecture. The cost of gradient-based optimisation scales with either the number of design variables or objectives and constraints. Whether this scaling is linear (thus poor) can heavily impact the computational efficiency of the optimisation framework. For instance, a non-intrusive method such as finite differences scales linearly with the number of design variables. For each new design variable the function has to be perturbed and compared to the non-perturbed function. Furthermore, finite difference schemes can return inaccurate derivatives if the design space has a noisy output function. Intrusive methods such as the direct and adjoint method scale with the number of inputs or outputs, respectively. Thus, the number of inputs or outputs can be increased without any additional computational cost. However, the implementation of these intrusive gradient assessment methods is non-trivial. The adjoint and direct method require derivative verification procedures. Derivative verification can be performed by comparing the model's derivatives to the derivatives returned by the complex-step method. The complex-step is accurate to machine precision in terms of accuracy. However, it is slow and scale linearly with the number of design variables. For these reasons, the complex-step method should only be used for derivative verification.\\
\\
The design methodology consists of three chapters that each contain \textit{model theory}, \textit{derivative verification} and \textit{model validation}. The three chapter describe the propeller, wing and slipstream model.

The propeller model uses a hybrid blade element momentum (HBEM) model called HELIX. HBEM is an adaptation of blade element momentum (BEM) theory that improves the classical BEM's versatility by including analyses for hover and forward flight (in a helicopter configuration) configurations. BEM calculates propeller performance by relying on blade element theory and momentum theory to converge to the same thrust coefficient values. HELIX uses a stall model to account for airfoil stall and transition. The input variables for the sectional airfoil lift are lift slope, zero lift angle of attack, and the stall angle. The stall model significantly improves the fidelity of the propeller model since it prevents the optimiser from exploiting non-physical angles of attack. The only modification that was made to the propeller was including a subsystem that returns the propeller's velocity distribution as this was required for the slipstream model. Although HELIX has been configured to optimise blade chord and sectional airfoil characteristics, it should be noted that including some of these variables in the design variable vector will return non-feasible results. For instance, it was seen that including propeller chord as a design variables would return a propeller with a maximised chord value near the root, whereas the remainder of the propeller chord would be at the lower bound. The model validation showed that the propeller code overestimates lift at higher advance ratios. The thrust over-estimation at higher advance ratios should be considered when interpreting the optimisation results and coupled model validation.\\
\\
The wing's aerostructural properties are modelled using the open source code OpenAeroStruct~\cite{Jasa2018a} (OAS). OAS uses a Vortex Lattice Method (VLM) to model the wing's performance and either a wingbox or tube model to predict structural performance. OAS also considers the wing deformation due to aerodynamic loads, and its effect on wing aerodynamics. After the wing is deformed, the updated wing aerodynamic performance is assessed. OAS iterates on this deformation until the system converges, after which the wing characteristics are returned. OAS required a number of modifications. The first being a remeshing function. The remeshing function is necessary since the propeller must align with the VLM panels. The remeshing function guarantees continuous propeller location and radius design variables. However, it is expected that the remeshing function introduces an error that is accumulating over the optimisation iterations. The second modification is important for coupling the wing to the slipstream model. The correction factor matrix returned by the slipstream model (discussed in the next paragraph) is added to the Aerodynamic Influence Coefficient (AIC) matrix. Lastly, the velocity vector, similarly returned by the slipstream model, has to be incorporated in the VLM system. It is important to note that any model modifications are accompanied by model derivative modifications as well.\\
\\
The propeller and wing model are coupled with a slipstream model. The slipstream model is based on the Rethorst correction factor~\cite{rethorst1958aerodynamic} and returns a correction factor matrix. The correction factor guarantees that the slipstream boundary conditions are satisfied. Without the correction factor the VLM would consider the slipstream to be of infinite height. In reality, the slipstream has a circular shape. The correction factor guarantees that the slipstream has this circular shape. The Rethorst correction factor was used in several wing-propeller analysis studies. Optimisation however introduces a number of challenges that do not have to be solved for analysis purposes. One of the requirements for optimisation is to have a system that is fully continuous. Additionally, the Rethorst correction factor requires the VLM panel to be in the centre of the propeller slipstream. Furthermore, the slipstream edges have to be aligned with the slipstream boundaries. For analysis, one could simply create a VLM mesh that satisfies both of these conditions. This is not possible for optimisation since it would require remeshing each iteration if the wing span, propeller radius or propeller location change. Remeshing introduces a highly discontinuous system that will cause the optimiser to diverge. To solve this issue, an overset mesh was configured for the correction factor. The overset mesh scales with the propeller radius and translates to the left and right with the propeller location. The overset mesh always covers the entire wing. The correction factor of the overset mesh is interpolated to the VLM mesh and neglects the values that are outside the wing span. This approach is similar to how wing-tip propellers are simulated in previous studies using the Rethorst correction factor. Together with the remeshing function in OAS this approach returns a correction factor and a velocity factor that are provided to OAS. Another important note about the Rethorst correction factor is that it consists of an even and odd solution. The odd solution is however computationally expensive due to the nested integrals with Bessel functions. Research by Nederlof~\cite{nederlof2020improved} shows that the even solution alone returns an acceptable solution. For these reasons the odd solution is neglected in the current slipstream model. The slipstream model does contain the code to include the odd solution. In future iterations of the framework a more efficient algorithm for Bessel function assessment~\cite{DEAmos_Algorithm644} should be used to increase the fidelity of the model. The model derivatives were verified by comparing to a finite differences scheme.\\
\\
Model coupling was done using OpenMDAO~\cite{Gray2019a}, an open-source optimisation frame well suited for gradient based optimisation. An eXtended Design Structure Matrix (XDSM) is given in \autoref{fig:XDSM_summary}. The propeller model passes the propeller velocity distribution to the slipstream model, that will return a correction matrix and velocity vector. The correction factor and velocity vector are included in the VLM system that assesses the aerodynamic performance of the wing. Finally, using the returned lift, drag, weight, and propeller power the constraints and objective functions can be calculated.

\begin{figure}[!ht]
    \centering
    \includegraphics[width=0.9\linewidth]{figures/05_modelcoupling/coupled_wingprop.pdf}
    \caption{The eXtended Design Structure Matrix for the wing-propeller system}
    \label{fig:XDSM_summary}
\end{figure}

The coupled wing-propeller optimisation returned a slower spinning propeller and the twist was adjusted such that the thrust constraint was satisfied, the isolated wing optimisation resulted in a twist distribution that was lowest at the tips and highest at the root. The twist did however slightly reduce behind the propeller. This is likely because the propellers have a very significant impact on the lift distribution. The propeller augmented wing section produces substantially more lift than the rest of the wing, even with smaller twist values. The optimisation results could be affected by modelling or numerical issues. However, the aim of this research was to show that coupled aerostructural wing-propeller optimisation is possible, which was achieved. Furthermore, it was noted that the optimiser tended to move the propellers inboard. The optimiser moved the propellers inboard due to the coupled model's inability to model tangential velocity components. Wing-tip propellers that rotate inboard-up could reduce the induced drag by increasing the effective aspect ratio~\cite{sinnige2018aerodynamic}. The effective aspect ratio increases by wing tip vortex dispersion, which can only be accounted for if swirl is included in the slipstream model.

The title of this dissertation is \textit{Towards Computationally Efficient Aerostructural Coupled Wing-Propeller Optimisation}. Therefore, the methodological results and recommendations are almost as important as the optimisation results. It was discovered that the remeshing function is prone to introduce asymmetries in the system. Asymmetries in the remeshing function are negligible for analysis purposes. However, the adjoint method uses partial derivatives. If the remeshing function is asymmetric it also introduces asymmetries in the differentiated code and could lead the optimiser to non-physical results (for instance moving one propeller inboard and the other outboard). The asymmetries in the model were resolved but any future modifications should take asymmetries between the left and right side of the wing into account. Future iterations of the framework could also include slipstream contraction and deflection, acoustic analysis, a model for tangential flow components and the odd Rethorst correction. It should also be noted that the current wing structural tube model should be replaced with a wingbox model.

The research project resulted a novel propeller wing optimisation framework. Future iterations of the framework can elaborate the framework and include other relevant aerodynamic, structural or acoustic models. The framework has its flaws and serves as a preliminary version that shows the feasibility of computationally efficient coupled mid-fidelity aerostructural wing-propeller optimisation.